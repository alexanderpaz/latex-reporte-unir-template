\documentclass[11pt, a4paper, oneside]{article}

% IDIOMA Y CODIFICACIÓN DE DOCUMENTOS 
\usepackage[spanish, es-tabla]{babel}
\usepackage[utf8]{inputenc}


% CONFIGURACIÓN DEL DISEÑO DE HOJA
\usepackage{geometry}
\geometry{a4paper, portrait, margin=1in}


% ESPACIO INTERLINEADO Y ENTRE PÁRRAFOS
\usepackage{setspace}
\linespread{1.5}
\setlength{\parskip}{1em}


% PARA GENERAR TEXTO ALEATORIO
\usepackage{lipsum}


% SETEO DE TIPOGRAFÍA
\usepackage{inconsolata}
\renewcommand{\familydefault}{\sfdefault}


% BIBLIOGRAFÍA
\usepackage[style=apa, backend=biber]{biblatex}
\addbibresource{referencias/referencias.bib}
\usepackage{csquotes}


% TO-DO NOTES
\usepackage{todonotes} %[disable]
\newcommand{\todocorregir}[2][]{\todo[color=red!70, #1]{#2}}
\newcommand{\todoverificar}[2][]{\todo[color=orange, #1]{#2}}
\newcommand{\todoidea}[2][]{\todo[color=green!20, #1]{#2}}
% uso:
%\todototoc{Esto se va directo a la lista de todos}
%\todocorregir[color=green!40]{Esto está mal}
%\missingfigure{Para recordar que hay que insertar una figura}


% DEFINICIÓN DE COLORES PERSONALIZADOS
\usepackage{xcolor}
\definecolor{colorunir}{rgb}{0,152,205}
\definecolor{codegreen}{rgb}{0,0.6,0}
\definecolor{codegray}{rgb}{0.5,0.5,0.5}
\definecolor{codepurple}{rgb}{0.58,0,0.82}
\definecolor{codeback}{RGB}{252,252,252}
\definecolor{codeborder}{RGB}{240,240,240}


% LISTADOS DE CÓDIGO
\usepackage{listingsutf8}
\usepackage{url}
\lstdefinestyle{reporte-style}{
	keywordstyle=\bfseries\ttfamily\color{codepurple},
	commentstyle=\color{codegray}\textit,
	stringstyle=\color{codegreen},
	backgroundcolor=\color{codeback},
	captionpos=b, 
	basicstyle=\ttfamily\footnotesize,
	frame={single},
	rulecolor=\color{codeborder},
	framesep=5px,
	breaklines=true,
	inputencoding=utf8,
	extendedchars=true,
	numbers=left,
	numbersep=1.5em,
	numberstyle=\ttfamily\small\color{codegray},
	title=\lstname
}
\input{configuracion/config-lst-utf.tex}
\lstset{style=reporte-style}


% CONFIGURACIÓN DE CAPTIONS
\usepackage{caption}
\captionsetup[figure]{name=Figura, labelsep=period} % Cambia el nombre a las figuras
\renewcommand\lstlistingname{Listado}
\renewcommand\lstlistoflistings{\textbf{{\Large Índice de listados}}}