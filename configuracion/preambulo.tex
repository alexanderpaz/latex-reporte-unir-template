\documentclass[11pt, a4paper, oneside]{article}

% IDIOMA Y CODIFICACIÓN DE DOCUMENTOS 
\usepackage[spanish, es-tabla]{babel}
\usepackage[utf8]{inputenc}


% CONFIGURACIÓN DEL DISEÑO DE HOJA
\usepackage{geometry}
\geometry{a4paper, portrait, margin=1in}


% ESPACIO INTERLINEADO Y ENTRE PÁRRAFOS
\usepackage{setspace}
\linespread{1.5}
\setlength{\parskip}{1em}


% PARA GENERAR TEXTO ALEATORIO
\usepackage{lipsum}


% SETEO DE TIPOGRAFÍA
\usepackage{inconsolata}
\renewcommand{\familydefault}{\sfdefault}


% BIBLIOGRAFÍA
\usepackage[style=apa, backend=biber]{biblatex}
\addbibresource{referencias/referencias.bib}
\usepackage{csquotes}


% TO-DO NOTES
\usepackage{todonotes} %[disable]
\newcommand{\todocorregir}[2][]{\todo[color=red!70, #1]{#2}}
\newcommand{\todoverificar}[2][]{\todo[color=orange, #1]{#2}}
\newcommand{\todoidea}[2][]{\todo[color=green!20, #1]{#2}}
% uso:
%\todototoc{Esto se va directo a la lista de todos}
%\todocorregir[color=green!40]{Esto está mal}
%\missingfigure{Para recordar que hay que insertar una figura}


% DEFINICIÓN DE COLORES PERSONALIZADOS
\usepackage{xcolor}
\definecolor{colorunir}{rgb}{0,152,205}
\definecolor{codegreen}{rgb}{0,0.6,0}
\definecolor{codegray}{rgb}{0.5,0.5,0.5}
\definecolor{codepurple}{rgb}{0.58,0,0.82}
\definecolor{codeback}{RGB}{252,252,252}
\definecolor{codeborder}{RGB}{240,240,240}


% LISTADOS DE CÓDIGO
\usepackage{listingsutf8}
\usepackage{url}
\lstdefinestyle{reporte-style}{
	keywordstyle=\bfseries\ttfamily\color{codepurple},
	commentstyle=\color{codegray}\textit,
	stringstyle=\color{codegreen},
	backgroundcolor=\color{codeback},
	captionpos=b, 
	basicstyle=\ttfamily\footnotesize,
	frame={single},
	rulecolor=\color{codeborder},
	framesep=5px,
	breaklines=true,
	inputencoding=utf8,
	extendedchars=true,
	numbers=left,
	numbersep=1.5em,
	numberstyle=\ttfamily\small\color{codegray},
	title=\lstname
}
\lstset{literate=
	{á}{{\'a}}1 {é}{{\'e}}1 {í}{{\'i}}1 {ó}{{\'o}}1 {ú}{{\'u}}1
	{Á}{{\'A}}1 {É}{{\'E}}1 {Í}{{\'I}}1 {Ó}{{\'O}}1 {Ú}{{\'U}}1
	{à}{{\`a}}1 {è}{{\`e}}1 {ì}{{\`i}}1 {ò}{{\`o}}1 {ù}{{\`u}}1
	{À}{{\`A}}1 {È}{{\'E}}1 {Ì}{{\`I}}1 {Ò}{{\`O}}1 {Ù}{{\`U}}1
	{ä}{{\"a}}1 {ë}{{\"e}}1 {ï}{{\"i}}1 {ö}{{\"o}}1 {ü}{{\"u}}1
	{Ä}{{\"A}}1 {Ë}{{\"E}}1 {Ï}{{\"I}}1 {Ö}{{\"O}}1 {Ü}{{\"U}}1
	{â}{{\^a}}1 {ê}{{\^e}}1 {î}{{\^i}}1 {ô}{{\^o}}1 {û}{{\^u}}1
	{Â}{{\^A}}1 {Ê}{{\^E}}1 {Î}{{\^I}}1 {Ô}{{\^O}}1 {Û}{{\^U}}1
	{ã}{{\~a}}1 {ẽ}{{\~e}}1 {ĩ}{{\~i}}1 {õ}{{\~o}}1 {ũ}{{\~u}}1
	{Ã}{{\~A}}1 {Ẽ}{{\~E}}1 {Ĩ}{{\~I}}1 {Õ}{{\~O}}1 {Ũ}{{\~U}}1
	{œ}{{\oe}}1 {Œ}{{\OE}}1 {æ}{{\ae}}1 {Æ}{{\AE}}1 {ß}{{\ss}}1
	{ű}{{\H{u}}}1 {Ű}{{\H{U}}}1 {ő}{{\H{o}}}1 {Ő}{{\H{O}}}1
	{ç}{{\c c}}1 {Ç}{{\c C}}1 {ø}{{\o}}1 {å}{{\r a}}1 {Å}{{\r A}}1
	{€}{{\euro}}1 {£}{{\pounds}}1 {«}{{\guillemotleft}}1
	{»}{{\guillemotright}}1 {ñ}{{\~n}}1 {Ñ}{{\~N}}1 {¿}{{?`}}1 {¡}{{!`}}1 
}
\lstset{style=reporte-style}


% CONFIGURACIÓN DE CAPTIONS
\usepackage{caption}
\captionsetup[figure]{name=Figura, labelsep=period} % Cambia el nombre a las figuras
\renewcommand\lstlistingname{Listado}
\renewcommand\lstlistoflistings{\textbf{{\Large Índice de listados}}}